\section*{Preface}
\markboth{Preface}{Preface}
\label{preface}

After the publishing of version 0.79.5, the project has been dormant for another half decade. As time has passed by, it is foreseeable that optical media will become extinct just like floppy discs did long ago. Still, it is important to preserve the contents of already existing optical media. Towards this end, we will maintain dvdisaster to keep it usable on current hardware and OS versions.

We do not plan for a rapid release cycle. Neither will many new features be introduduced like it was at the beginning of the project. Instead, we will strive to fill in the missing gaps left over from the still open RS03 release cycle. Afterwards, dvdisaster shall be kept sound and complete as long as optical media are still in use.

\bigskip

{\em -- -- The dvdisaster development team, Spring 2021}


\newpage
\section*{Preface for version 0.79.5}

Since the release of dvdisaster 0.79.3\footnote{Version 0.79.4 was never finished
and released.}, nearly five years have passed.
This was partly due to changed circumstances in its
primary developer's life, but there was also a lot of
coding going on behind the scenes. In comparison with its
predecessor, dvdisaster 0.79.5 comes with lots of its
internals being significantly reworked.

\smallskip

The most visible improvement  of dvdisaster 0.79.5 is, of
course, its multithreaded RS03 codec. While it takes
about 62 minutes for protecting a 36 GiB image with RS02
on a mid range PC,
the same task is done with RS03 in less than 7 minutes
using 6 processor cores on the same machine.
On a high end server with at least 16 cores and very good I/O,
this can be done in under a minute. That's quite an
improvement.

RS03 is ready for production use in the current release.
Some non-essential features, especially reworking the
adaptive reading for use with RS03 and multi-threaded
RS03 decoding (media fixing) will be delivered with
the following dvdisaster releases.

\smallskip

Other parts of the project had to be changed or even
discontinued. A software project lives on development
and continuous releases; else the
project will eventually die. In this respect, dvdisaster
was very endangered in the last few years.
To prevent this from happening again, most effort
is now directed into source code development;
everything else is delegated or discontinued.
Source code development basically means making
the GNU/Linux version, which provides the code base
for all other versions, and the FreeBSD and NetBSD ports,
which are very easily derived from the GNU/Linux code.
This is not the case for the Mac OS and Windows ports,
which are, unfortunately, \tlnk{qa-discontinued-os}{discontinued} as of now.

Another feature which has to go are the separate
stable and development releases.
Starting with this version, all dvdisaster releases
are considered production quality, so there is no
need for different branches anymore.

\smallskip

Maintaining the multi-lingual online documentation, which
also served as the project home page, did also prove to
be too time consuming. The project home page has
been changed into a simple download platform for
the project sources. It is now directed at package
maintainers who will create and pass on binaries
for the GNU/Linux, FreeBSD and NetBSD distributions.

The program documentation, which you are reading
right now, is provided in PDF format which is much
easier to author than the HTML version. The only
language available is English. Most parts of this
manual have been adapted from the old online
documentation, so it still feels more like a website
than a book. While hyperlinks are not as usable in PDF
as in HTML, they have been kept in this document to
stress that it is intended to be used as an online reference.
So please do our environment a favour and do not print
this manual. It is not meant to be read
from front cover to back cover, anyways.

\smallskip

Okay, enough ranting already. May dvdisaster be helpful
in protecting and recovering your valuable data,
and thanks for using it!

\bigskip

{\em -- -- cg, August 2015}


