\section{Downloading dvdisaster}
\label{download}

\begin{tabular}{ll}
\hspace*{-2mm}\begin{minipage}{122mm}
dvdisaster is available for \tlnk{download-requirements}{recent versions} of the FreeBSD,
GNU/Linux and NetBSD operating systems. It is provided
as \href{http://fsfe.org/about/basics/freesoftware.en.html}{free Software} under
the \href{http://www.gnu.org/licenses/gpl-3.0.txt}{GNU General Public License v3}.
\end{minipage} &
\begin{minipage}{34mm}
\includegraphics[width=34mm]{icons/gplv3-127x51.png}
\end{minipage}
\end{tabular}

\bigskip

The dvdisaster developer site (\url{http://dvdisaster.net}) contains
the latest source code releases for the FreeBSD, GNU/Linux and NetBSD
operating systems. These are mostly aimed at maintainers of binary packages for
the beforementioned platforms. As an end user you might find it more convenient
to install dvdisaster from the package system of your operating system bundle or
distribution. But if you prefer to download and compile the source package on your
own, you're welcome to visit the site.

\bigskip

The source code archives contain a file {\tt INSTALL} with further instructions
for building dvdisaster. The build process follows the 
common {\tt ./configure; make; make install} scheme. The installation step
is optional; you can use dvdisaster directly from the build tree.

\subsection{System requirements}
\label{download-requirements}

\paragraph{Hardware requirements}\quad

\begin{itemize}
\item x86, PowerPC or Sparc processor;
\item an up-to-date CD/DVD/BD drive with ATAPI, SATA or SCSI interface;
\item enough hard disk space for creating .iso images from processed media.
\end{itemize}

\bigskip

{\bf Supported operating systems}\quad

\medskip

The following table gives an overview of the supported operating systems.
The specified releases have been used for developing and testing the
dvdisaster version made available as of this writing. 
Typically, slightly older and newer OS versions will also work.
\label{download-requirements-freebsd}

\bigskip

\begin{tabular}{|l|l|}
\hline
Operating System & Release \\
\hline
GNU/Linux & Debian Jessie 8.0,  Kernel 3.16 \\
\hline
FreeBSD & 10.1 \\
\hline
NetBSD & 6.1.5 \\
\hline
\end{tabular}

\bigskip

\paragraph{Not supported operating systems}\quad

\medskip

Support for Windows and Mac OS has been ended
and is not planned to be resumed in the
future (see \tlnk{qa-discontinued-os}{QA item 2.4 for an explanation)}. 

\newpage
\subsection{Make sure you're not getting ripped off: The small print (and other things).}
\label{download-terms}

The dvdisaster project provides this software
as \href{http://fsfe.org/about/basics/freesoftware.en.html}{free software} to you using
the \href{http://www.gnu.org/licenses/gpl-3.0.txt}{GNU General Public License v3}.
The dvdisaster project also wants to make sure that you know you can download
the software at no cost and keeping your full privacy.
To make it clear how we distribute dvdisaster, what we do and what we
won't do, we have compiled the following list:

\paragraph{Internet and download sites}\quad

\smallskip

The dvdisaster project uses the following internet domains for publishing
its web sites and supplying software downloads:

\begin{center}
  \begin{tabular}{l}
dvdisaster.com\\
dvdisaster.de\\
dvdisaster.net\\
dvdisaster.org
  \end{tabular}
\end{center}

All domains are forwarded to the same site at dvdisaster.net.
No other internet or download sites are run by the dvdisaster project.

\paragraph{No money or personal data required.}\quad

\smallskip

There is {\bf no registration process} for using this software.
The dvdisaster project {\bf never} asks you to enter personal data,
to pay a fee or to donate money for:

\smallskip

\begin{tabular}{ll}
$\bullet$ & using its internet and download sites,\\
$\bullet$ & downloading the software, and\\
$\bullet$ & running the software. \\
\end{tabular}

\smallskip

In fact, please do not offer any donations to the project. 
We cannot accept them for various reasons.

\paragraph{Cryptographic signature and checksums}\quad
\smallskip

dvdisaster releases are always published with cryptographic signatures
and md5 checksums. See the \href{http://dvdisaster.net}{download site} for examples.
Be very cautious if signatures and checksums are missing, invalid or not
matching those published at the sites mentioned above.
