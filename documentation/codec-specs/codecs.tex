\documentclass[12pt,a4paper,twoside]{article}
\usepackage{times}	% Skalierbarer und lesbarer Zeichensatz
\usepackage{ucs}	% Benötigt für Eingabe von unicode-Zeichensätzen
\usepackage[utf8x]{inputenc} % Aktiviert Eingabe von unicode-Zeichensätzen
\usepackage{epsfig}	% Makros zum Einfügen von Grafiken
\usepackage{anysize}	% Makros zum Einstellen der Seitenränder
%\usepackage{makeidx}	% Makros zum Erstellen des Indexes
\usepackage{url}
\usepackage{hyperref}
\usepackage{fancyhdr}
\usepackage{listings}
\usepackage{color}
\usepackage{comment}

\marginsize{30mm}{20mm}{20mm}{20mm}  % Seitenränder links, rechts, oben, unten
\parindent0em		% Keine amerikanische Einrückung am Anfang von Paragraphen
\renewcommand{\floatpagefraction}{.99}

\pagestyle{fancy}	% Seitenstil
%\makeindex		% wird für Erstellung von Stichwortverzeichnissen benötigt

% Ende der Voreinstellungen

\newcommand{\paperversion}{{\em Version 1.01}}

\fancyhead{}
\fancyhead[LE,RO]{page \thepage\ of \pageref{LastPage}}
\fancyhead[LO,RE]{\nouppercase{\rightmark}}
\renewcommand{\sectionmark}[1]{\markboth{#1}{#1}}
\renewcommand{\subsectionmark}[1]{}

\renewcommand{\footrulewidth}{0.4pt}
\fancyfoot{}
\fancyfoot[LE,RO]{dvdisaster codec specification}
\fancyfoot[LO]{created: \today}
\fancyfoot[RE]{\paperversion}

\input{../config/version.tex}

\begin{document}

\definecolor{lightblue}{RGB}{224,224,255}
\pagecolor{lightblue}
\title{The dvdisaster Reed-Solomon Codec specification}
\author{The dvdisaster development team\\support@dvdisaster.org}
\date{}
\maketitle
\thispagestyle{empty}

\centerline{\includegraphics[width=32mm]{title.eps}}
  
\begin{center}
\paperversion
\end{center}

\bigskip

\begin{abstract}
This is the {\em blue manual}, describing the data formats of the dvdisaster
Reed-Solomon codecs which are currently called RS01, RS02 and RS03.
The codecs create Reed-Solomon parity data to protect data stored on optical media.
Depending on the codec, parity data can either be stored in a separate file 
or be integrated with the .iso image on the same medium. 
See \homepage for additional resources on the dvdisaster pro\-ject,
e.g. for the {\em orange manual} (manual.pdf) containing information for end users.
\end{abstract}

\bigskip

{\bf Target audience.} This paper is primarily intended as a working base for the
dvdisaster developers and, when the final version has been crafted, as an implementation
guide for third party developers who wish to create and process dvdisaster error correction data.
It is {\bf neither intended nor suitable} as end-user documentation; for usage information
please refer to the online documentation at \homepage.

\bigskip

{\bf Prerequisites.} This paper assumes profound knowledge of coding theory and the 
underlying math. The reader is assumed to have a thorough understanding of Reed-Solomon
codes, both in theory and from an implementation viewpoint. A basic understanding
of programming in C is also assumed.

\vfill
\begin{center}
{\em 
Copyright 2008-2015 Carsten Gnörlich.\\
Copyright 2021 The dvdisaster development team.\\
Verbatim copying and distribution of this entire article is permitted in any medium, 
provided this notice is preserved.}
\end{center}

\newpage
\nopagecolor

% Changelog

\input{changelog}

% Table of Contents

\tableofcontents
\newpage

% Reed-Solomon encoding details

\input{rs03}

\input{rs02}

\input{rs01}

% Header formats

\appendix

\input{ecc-header}

\input{crc-block}

\input{rs-params}
\end{document}
